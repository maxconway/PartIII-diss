% some equations might make this look a bit more pro
Flux Balance Analysis~\cite{Orth2010} is a technique used to predict the average rates of reactions in an organism's metabalome, given its stochiometric matrix.
It relies on two assumptions:
\begin{enumerate}
\item \label{ass:steadystate} The metabolome of the organism under study is in a steady state.
\item \label{ass:maxgrowth} The organism produces maximum possible growth for any given combination of resources available and physical constraints.
\end{enumerate}
Assumption \ref{ass:steadystate} allows the system of simultaneous differential equations describing the metaboloism to be reduced to a set of simultaneous algebraic equations, since in steady state we do not expect any net accumulation of metabolites. 
This limits possible rates to the interior of a convex polytope. 
To predict the actual position within this polytope, assumption \ref{ass:maxgrowth} is used, and an optimization technique, such as linear programming, is used to find the point, or space, representing highest growth.