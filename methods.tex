\chapter{Methods}

\section{Overview of Genetic Design Techniques Studied}

\subsection{Genetic Design by Local Search (GDLS)}
%this all needs some nicer presentation
Genetic Design through Local Search~\cite{Lun2009}, or GDLS is an algorithm that is used to engineer organisms that produce particular metabolites. 
Given an FBA model with gene-protein reaction annotations, it produces a knockout strategy which will alter the metabolic network to overproduce the compound required.

GDLS is a local search algorithm, which means that it starts with a solution and recursively improves upon it. 
This is a sensible decision in metabolic engineering, due to the natural starting point of no knockouts. 
The detailed procedure is as follows:

\begin{algorithm}
\caption{Pseudocode of Genetic Design by Local Search}
\label{alg:GDLS}
\begin{algorithmic}
\State{\(X_{0} \gets \text{natural}\)}
\While{\(X_{i} \text{is better than} X_{i-1}\)}\State{
	\(X_{i+1} \gets \text{neighbourhood\_search} X_{i} \)
}
\EndWhile
\end{algorithmic}
\end{algorithm}
%need to expand this

This means that at each iteration, the algorithm starts with a candidate genome, or genomes. For each of these, it evaluates every genome that differs by at most neighbourhoodsize knockouts or knockins, and selects the best from these. The search then starts again from these best solutions, and this is repeated until no better solutions are found. 

\subsection{Genetic Design by Multi-objective Optimization(GDMO)} 
Genetic Design by Multi-objective Optimization \todo{who it's written by} is a genetic design algorithm based on the multi-objective evolutionary optimization algorithm NSGA-II.
Using a multi-objective optimization algorithm in this context has a number of advantages.
\begin{itemize}
\item A trade-off between different objectives does not need to be selected before use. This avoids the issue of having to guess the properties of a system that has not yet been designed.
\item The Pareto front can be used to visualize the space of possibilities. This can provide insight into the different structures created by different pathways, and help to design an organism that will not suffer from reduced production due to evolution once in use.
\item Crucially, this approach \todo{slower or faster?} does not have to sacrifice performance for these advantages, since maintaining a population of solutions that occupies a wide range of good phenotypes in fact helps the evolutionary process.
\end{itemize}

\section{Computational Complexity analysis}
The core problem of this kind of genetic design (which does not use network structure information) is 0-1 integer linear programming~\cite{Karp1972}---indeed GDLS works by repeatedly solving small cases of this problem.
This means that this problem is NP-complete, and assuming \(\mathcal{P} \neq \mathcal{NP}\), these algorithms have running times that are not bounded by any polynomial.
This suggests extremely long running times for some combinations of input parameters.
This makes it important to be able to form an \emph{a priori} estimate of the running time, so that computation time is not wasted with infeasibly long runs.
These very long running times mean that forming a model purely by regression analysis over the whole parameter space would be impractical, and so a partly analytic approach must be taken, by examining the algorithm's procedures.

\subsection{Flux Balance Analysis}
Since FBA is at the core of the algorithms studied, it makes sense to consider this separately. 
Here, FBA is done via the Simplex linear programming algorithm, which has which takes \(O(n^3)\) steps for practical problems~\cite{Dantzig1963}. 
%A number of different solvers exist, with GLPK being used here, as the lowest common denominator.
%solvers vary in code speed by around 100-fold
In order to characterize the differences between running times for different models, FBA was performed 100 times with each or a number of models. The results, shown in figure~\ref{fig:FBAtimings}, show that times can vary by \todo{find out fold change} X-fold. This can be attributed to the variance in sizes of the different models used.

\begin{figure}
\label{fig:FBAtimings}
\includegraphics[width=\textwidth]{./data/FBAtimings.png}
%image may need title adjustment
\caption{running times for FBA of metabolic models}
\end{figure}

\subsection{GDMO}
Since GDMO is an evolutionary algorithm, the number of objective function evaluations performed is proportional to the population size multiplied by the number of generations evaluated. 
Each objective evaluation requires that Flux Balance Analysis is conducted a constant number of times. 
This means that the total runtime can be modelled can be modelled via in R as the interaction of the three quantities of iterations, population and time to conduct one round of FBA.

%This is a time for a bit of sweave
%actually do stats

\subsection{GDLS}
GDLS does not run to a hard limit, but instead stops when it finds the local optimum, so we cannot work out a useful upper bound on running time. 
A lower bound is, however, possible, and appears from exploratory experiments to be useful.
Four variables are used here to parametrize GDLS: 
\begin{enumerate}
\item The metabolic model used, let us call the number of potential knockouts in it \texttt{modelsize};
\item \texttt{nbhdsz}, number of knockouts distance from a solution that is exhaustively searched;
\item \texttt{M}, the number of search paths to maintain at each stage; and
\item \texttt{knockouts}, the maximum total number of knockouts (fixed at 40 for fair comparison with GDMO).
\end{enumerate}
Here we can see that the metabolic model used will have a very large effect, since not only does this effect the time to complete one FBA evaluation, but the space that much be searched around a given solution is \(\texttt{modelsize}^\texttt{neighbourhoodsize}\). This means that, total search time for one solution is roughly \(\texttt{modelsize}^(3+\texttt{neighbourhoodsize}\), once the time to conduct FBA is included. Overall, this gives a lower bound on running time of \(\texttt{M} * \texttt{modelsize} ^ ( 3 + \texttt{neighbourhoodsize}\).
\todo{is branching factor number of search paths, or is it at each stage}

\section{Instrumentation}
To facilitate comparison, the genetic design strategies were instrumented in order to record every candidate solution that they produced. This allows the improvement of the candidates over time to be seen. 
Tests were conducted to ensure that this instrumentation code did not influence the results by its own execution time; it was found to run fast enough that the slowdown was undetectable.
