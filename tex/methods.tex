\chapter{Methods}

\section{Techniques Studied}

\subsection{Genetic Design by Local Search (GDLS)}
%this all needs some nicer presentation
Genetic Design through Local Search~\cite{Lun2009}, or GDLS is an algorithm that is used to engineer organisms that produce particular metabolites. 
Given an FBA model with gene-protein reaction annotations, it produces a knockout strategy which will alter the metabolic network to overproduce the compound required.

GDLS is a local search algorithm, which means that it starts with a solution and recursively improves upon it. 
This is a sensible decision in metabolic engineering, due to the natural starting point of no knockouts. 
The detailed procedure is as follows:

\begin{algorithm}
\caption{Pseudocode of Genetic Design by Local Search}
\label{alg:GDLS}
\begin{algorithmic}
\State{\(X_{0} \gets \text{natural}\)}
\While{\(X_{i} \text{is better than} X_{i-1}\)}\State{
	\(X_{i+1} \gets \text{neighbourhood\_search} X_{i} \)
}
\EndWhile
\end{algorithmic}
\end{algorithm}
%need to expand this

This means that at each iteration, the algorithm starts with a candidate genome, or genomes. For each of these, it evaluates every genome that differs by at most neighbourhoodsize knockouts or knockins, and selects the best from these. The search then starts again from these best solutions, and this is repeated until no better solutions are found. 

\subsection{Genetic Design by Multi-objective Optimization(GDMO)} 

