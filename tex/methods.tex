\chapter{Methods}

\section{Overview of Genetic Design Techniques Studied}

\subsection{Genetic Design by Local Search (GDLS)}
%this all needs some nicer presentation
Genetic Design through Local Search~\cite{Lun2009}, or GDLS is an algorithm that is used to engineer organisms that produce particular metabolites. 
Given an FBA model with gene-protein reaction annotations, it produces a knockout strategy which will alter the metabolic network to overproduce the compound required.

GDLS is a local search algorithm, which means that it starts with a solution and recursively improves upon it. 
This is a sensible decision in metabolic engineering, due to the natural starting point of no knockouts. 
The detailed procedure is as follows:

\begin{algorithm}
\caption{Pseudocode of Genetic Design by Local Search}
\label{alg:GDLS}
\begin{algorithmic}
\State{\(X_{0} \gets \text{natural}\)}
\While{\(X_{i} \text{is better than} X_{i-1}\)}\State{
	\(X_{i+1} \gets \text{neighbourhood\_search} X_{i} \)
}
\EndWhile
\end{algorithmic}
\end{algorithm}
%need to expand this

This means that at each iteration, the algorithm starts with a candidate genome, or genomes. For each of these, it evaluates every genome that differs by at most neighbourhoodsize knockouts or knockins, and selects the best from these. The search then starts again from these best solutions, and this is repeated until no better solutions are found. 

\subsection{Genetic Design by Multi-objective Optimization(GDMO)} 

\section{Computational Complexity analysis}
The core problem of this kind of genetic design (which does not use network structure information) is 0-1 integer linear programming~\cite{Karp1972}---indeed GDLS works by repeatedly solving small cases of this problem.
This means that this problem is NP-complete, and assuming \(\P \neq \NP\), these algorithms have running times that are not bounded by any polynomial.
This suggests extremely long running times for some combinations of input parameters.
This makes it important to be able to form an \emph{a priori} estimate of the running time, so that computation time is not wasted with infeasibly long runs.
These very long running times mean that forming a model purely by regression analysis over the whole parameter space would be impractical, and so a partly analytic approach must be taken, by examining the algorithm's procedures.

\subsection{Flux Balance Analysis}

\subsection{GDMO}
Since GDMO is an evolutionary algorithm, the number of objective function evaluations performed is proportional to the population size multiplied by the number of generations evaluated.
Each objective evaluation requires that Flux Balance Analysis is conducted a constant number of times, via the Simplex linear programming algorithm, which has which takes \(O(n^3)\) steps for practical problems~\cite{
Dantzig1963}.
%actually do stats

\subsection{GDLS}
Since GDLS does not run to a hard limit, but instead stops when it finds the local optimum, it is most useful to first characterize the time complexity of a single iteration.