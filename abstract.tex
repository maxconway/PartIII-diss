% !TeX root = dissertation.Rnw    
This dissertation describes the use of genetic design algorithms to engineer strains of the bacteria \textit{Geobacter sulfurreducens} and \textit{metallireducens} which exhibit increased ability to produce electricity. 
A maximum electricity production of \SI{104}{\percent} of the wild type value was achieved.
In addition, these algorithms were used on \textit{Escherichia coli} to produce a strain with \SI{260}{\percent} acetate production gain over wild type.

With the help of data from the \href{http://www.geobacter.org/}{Lovley Lab}, at the University of Massachusetts, the biology of these strains was examined, to identify genes and reactions of special interest. These results include the discovery of a massive (20-fold) peak in genes related to metal reduction in the \textit{Geobacter metallireducens} chromosome.

In addition to the creation and evaluation of \num{1029480} \textit{Geobacter}, the properties of the genetic design algorithms themselves are investigated, alonge with the potential use of genetic design algorithms for knockins.